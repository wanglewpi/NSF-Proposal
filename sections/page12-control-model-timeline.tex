page12

Automotive Control Model 
Project timeline


\subsection{Broader Impacts}
Research into connected and autonomous vehicles is important.  With several major automotive 
companies pursuing research and development activities that will ultimately result in having 
road vehicles drive themselves, it is only a matter of time before this idea becomes a 
reality.  However, as with many other transformative technologies, there are numerous issues 
that were not initially considered that require immediate resolution.  In the case of future 
automotive systems, there are many such issues that could directly impact the personal safety 
of individuals, commerce/trade, recreational activities, public safety operations, and other 
activities dependent on reliable vehicular transportation.  Consequently, investigation of 
these cyber physical systems with respect to wireless connectivity and autonomy is essential 
for guaranteeing their continued operation. This project will make an educational 
contribution by mentoring and training at least two graduate students with an emphasis on 
identifying qualified students from underrepresented groups.  In addition to research 
findings resulting from this proposed effort being ultimately shared with the rest of the 
community, the outcomes of this project will also be employed as a teaching resource for 
undergraduate and graduate courses, as well as for K-12 educational outreach programs 
focusing on STEM. It is expected that this project will yield high impact, transformative 
research, which will be disseminated via the normal process of publication in archival 
journals and conferences.  Moreover, tutorials are planned for major conferences in CPS, 
controls, automotive technology, and communications.

WPI is renowned for its innovative and unique approach to engineering education, which is 
based on a project-oriented program emphasizing intensive learning experiences and the direct 
application of knowledge. With all the investigators being WPI faculty members, they all 
understand the interdependency between teaching and research and how each activity 
strengthens the other. Three primary goals of this proposal with respect to the educational 
plan will be described, namely: curriculum development, undergraduate involvement in the 
research activities, and graduate student mentoring. Furthermore, we will highlight how the 
knowledge and results obtained from the proposed facility can be directly integrated into the 
educational goals.
%
\begin{enumerate}[\textbullet]
	\listformat
	\item \textbf{Course Development --} This integration of the resulting outcomes from this 
	project with the curriculum will be achieved via the revision of seven existing courses 
	by introducing well-defined course ``modules'' based on this proposed facility.
	
	\item \textbf{Undergraduate Involvement of Research Activities  --} Another education 
	contribution resulting from this project is the mentoring of several senior-level 
	undergraduate design experiences related to the proposed activities during the academic 
	year.  These experiences, referred to as Major Qualifying Projects (MQPs), are a 
	quintessential component of the WPI Plan that every student must complete sometime during 
	their studies at WPI. The MQP can be conducted by either a single undergraduate student 
	or a group of students, and focuses on solving real-world problems using engineering 
	principles and skills.  Weekly meeting will be held to ensure that the project stays on 
	track with respect to specified milestones and objectives.
	
	\item \textbf{Undergraduate Research Assistantships --} In addition to the MQPs, this 
	proposed project will make an educational contribution by supporting, educating, and 
	training several undergraduate research assistants per year, including via the NSF 
	Research Experiences for Undergraduates (REU).  Moreover, Co-PI Wyglinski is currently 
	the faculty advisor for the \textit{WPI Women in Electrical and Computer Engineering} 
	(WECE), as well as a member of the \textit{IEEE Women in Engineering} (WIE), which are 
	organizations designed to foster the professional and personal development of female ECE 
	students. The PI and the Co-PI will actively encourage and solicit the participation of 
	females and other underrepresented groups.
	%
	\item \textbf{Graduate Student Supervision and Mentoring --} This project will make an 
	educational contribution by supporting, educating, and training one graduate student 
	research assistant per year over five years, and the research outcomes by this student 
	will constitute several parts of his/her degree requirements, e.g., thesis and/or 
	dissertation. 
	%
	\item \textbf{K-12 Outreach --} To encourage the advancement of STEM topics across local 
	area 
	high schools, the investigators will employ the proposed test-bed as an educational 
	platform.  Leveraging existing activities with Worcester Public Schools, as well as the 
	\textit{Massachusetts Academy of Mathematics and Science at WPI} (Mass Academy), several 
	classroom 
	modules conforming to the \textit{Massachusetts Mathematics Curriculum Framework} and the 
	\textit{Massachusetts Science and Technology/Engineering Curriculum Framework} will be 
	created and 
	taught at participating high schools by the investigators.  Moreover, mathematics and 
	science classes from local area schools, such as Mass Academy and Doherty Memorial High 
	School, will also be invited to make field trips to the proposed facility and participate 
	in several ready-made experiments.
\end{enumerate}
